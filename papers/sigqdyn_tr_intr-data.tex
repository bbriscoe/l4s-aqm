% !TeX root = sigqdyn_tr.tex
% ================================================================
\section{Introduction and Scope}\label{sigqdyntr_intro}

Much attention has been paid to reducing the delay experienced on the data path through packet networks. For instance, see sections II and IV of the extensive survey of latency reducing techniques in \cite{Briscoe14b:latency_survey}, which aim to reduce propagation delay, queuing delay, serialization delay, switching delay, medium acquisition delay and link error recovery delay.
%
%Propagation and queuing delay are typically the largest contributors to the overall delay experienced by network data.\footnote{For radio links medium acquisition delay is also highly significant.} Propagation delay can be reduced by structural techniques, such as server placement, but queuing delay is a result of subtle interactions due to the system design. 

This memo focuses on reducing unnecessary delays in the control system that attempts to match load to capacity. Reducing queuing delay has also been the focus of much other recent work. But the focus here is on cutting delays in the control path rather than the data path. That is, delays in measuring queue dynamics and in communicating the resulting control signals.

Once congestion signals are delayed, regulation of the load becomes more sloppy, and the queue tends to overshoot and undershoot more as a result, leading the data itself to experience greater peaks in queuing delay as well as intermittent under-utilization. And, perhaps most importantly, if the congestion signals due to bursts of data are delayed, even slightly, they will be applied to the packets just after each burst. Then bursty traffic could shift much of the blame for congestion onto other traffic running more smoothly in the background.

This memo applies to so-called immediate AQMs that intend to signal growth in the queue immediately. It is intended to ensure that packets carry signals that apply to those packets themselves, not to earlier packets, not even slightly earlier. Then signals intended to be applied to packet bursts will not be applied to packets after the burst. So-called Classic AQMs are not in scope because they deliberately filter out rapid variations in the queue before signalling more persistent queue growth, which introduces perhaps 100\,ms of smoothing delay.

To be concrete, this memo assumes congestion signals that are transmitted from an active queue management (AQM) algorithm~\cite{Adams13:AQM_survey} using either drop or explicit congestion notification (ECN)~\cite{Floyd94:ECN}, which are the only standardized signalling protocols~\cite{IETF_RFC3168:ECN_IP_TCP} for end-to-end use over one of the two Internet protocols, IPv4 and IPv6. The L4S ECN protocol~\cite{Briscoe15f:ecn-l4s-id_ID} is an experimental update to the standard ECN protocol intended to be used for immediate signalling of congestion. But the problems in this memo apply equally if an immediate AQM uses a different signalling protocol, such as the SCE~\cite{Morton19:SCE_ID} proposal. Most of the ideas concern only algorithmic improvements, which could be applied in other settings with completely different congestion signalling protocols.

\begin{figure*}
	\centering
	\includegraphics[width=0.8\linewidth]{sojourn-prob}
	\caption{Schematic Illustrating Two Problems with the Sojourn Time Metric. a) It does not measure the full size of a burst until the end (left); b) It does not measure a draining queue (right). Draining is visualized at one equisized packet per timeslot. Sojourn time is represented just before each packet is dequeued as the number of timeslots along its diagonal path.}\label{fig:sojourn-prob}
\end{figure*}

Control path delay consists of the following elements:
\begin{itemize}[nosep]
	\item propagation delay (in common with the data);
	\item queuing delay (in common with the data);
	\item measurement delay: measuring the queue, as well as arrival and/or departure rates;
	\item smoothing delay: filtering out fluctuations in measurements;
	\item signal encoding delay: a number representing the signal is produced within an AQM algorithm, which is then compressed into a unary `encoding' in each packet, and `decoded' by the congestion control algorithm's response to the unary-encoded signals. A unary encoding is used so that the AQM does not have to recognize flows or hold per-flow state. But it constrains the bandwidth of the signalling channel, which introduces encoding delay;
	\item randomization delay: randomness is introduced to break up oscillations, but it requires longer to detect the underlying signal.
\end{itemize}

This memo pays most attention to two of these: measurement and queuing delay. The other four are briefly surveyed in \S\,\ref{sec:related}.
