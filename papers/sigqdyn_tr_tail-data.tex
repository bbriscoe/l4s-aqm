% !TeX root = sigqdyn_tr.tex
% ================================================================
\section{Related Work}\label{sec:related}

Scaling sojourn time seems superficially similar to combined enqueue and dequeue ECN marking (CEDM)~\cite{Shan17:CEDM}, because CEDM marks a packet at enqueue if the queue is over a threshold, but then unmarks it at dequeue if the backlog has dropped below the threshold. However the two are significantly different. Firstly, CEDM has to be based on queue length in order to mark at enqueue. But also CEDM is intentionally asymmetric, in that it unmarks packets if the backlog at dequeue has dropped below the threshold, but it does not mark packets at dequeue if they have risen above the threshold. In contrast, scaling sojourn time is deliberately symmetric, meaning it compensates for growth or shrinkage of the backlog (\autoref{fig:scaled-sojourn}).

\section{Further Work}

These ideas might not be novel, but no concerted effort has been made to search the literature. The ideas have not been evaluated either.

\section{Acknowledgements}\label{sigqdyntr_acks}

%The breakdown of the elements of signalling delay in the Introduction is based on unpublished work for the author's PhD from 2006--07 entitled ``Necessary Changes Towards a Sufficient Internetwork Protocol;
%Simple, Scalable, Secure \& Responsive Resource Control''. 
The scaling of the service time of the queue was based on discussions with Henrik Steen, an MSc student of the author, in Nov 2016 \& May 2017.