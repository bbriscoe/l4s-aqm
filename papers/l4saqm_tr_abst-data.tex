% !TeX root = l4saqm_tr.tex
% ================================================================
This memo focuses solely on the native AQM of Low Latency Low Loss Scalable throughput (L4S) traffic and proposes various improvements to the original step design. One motivation for DCTCP to use simple step marking was that it was possible to deploy it by merely configuring the RED implementations in existing hardware, albeit in an unexpected configuration by setting parameters to degenerate values. However, there is no longer any imperative to stick with the original DCTCP step-function design, because changes will be needed to implement the DualQ Coupled AQM anyway, and the requirements for L4S congestion controls are not yet set in stone either. This paper proposes gradient (ramp) marking and a virtual (a.k.a.\ phantom) queue. It provides a way to implement virtual queuing delay (a.k.a.\ virtual sojourn time) and scaled virtual sojourn time.