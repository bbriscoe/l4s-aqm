% !TeX root = sigqdyn_tr.tex
% ================================================================
This paper focuses on reducing the delay before an active queue management (AQM) algorithm emits congestion signals, e.g.\ explicit congestion notification (ECN) or packet drop. These algorithmic delays can be greater than the delay that the data itself experiences within the queue. Once the congestion signals are delayed, regulation of the load becomes more sloppy, and the queue tends to overshoot and undershoot more as a result, leading the data itself to experience greater peaks in queuing delay as well as intermittent under-utilization. Also, even if an AQM algorithm only slightly delays its congestion signals, it tends to shift the signals from more bursty flows onto other smoother flows.

The importance of immediate congestion signalling has been recognized in approaches such as Data Center TCP (DCTCP) and Low Latency Low Loss Scalable throughput (L4S). However, this paper points out that the sojourn time metric that is increasingly used in these approaches introduces unnecessary internal measurement delay, which is worst during bursts. Expected service time is proposed as an alternative metric. Three potential implementations are proposed, which keep some or all of the benefits of sojourn time, but without the internal measurement delay. The paper also briefly surveys ways to remove other delays within AQMs, such as the delay due to randomness in the signal encoding.



